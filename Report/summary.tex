With the advancement of game development technology, 
procedural generation techniques are playing an increasingly crucial role in creating virtual environments, 
particularly in the fields of terrain generation and cave simulation. 
Utilizing mathematical algorithms and noise functions, 
procedural techniques can achieve highly variable environments, 
offering players experiences filled with exploration and novelty in each game session. 
Voxel representation is pivotal in the storage and depiction of terrain, 
serving not only to describe the geometric structure but also to facilitate real-time rendering. 
Proper voxelization processes effectively optimize scene performance.

In the context of cave generation, 
noise-based techniques simulate the intricate natural structures of underground spaces through the weighted combination of multiple noise functions. 
This approach enables the automatic generation of cave systems that exhibit diversity and realism.

This paper delves into the methodology and implementation of procedurally generating infinite voxel caves based on weighted multi-noise. 
By optimizing combinations of various noise functions, 
it achieves heightened naturalism and visual fidelity in terrain generation. 
Additionally, performance optimization is achieved through surface extraction techniques, 
facilitating dynamic loading and unloading of voxel worlds using block-based data structures.