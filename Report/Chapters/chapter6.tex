\chapter{Conclusions and Future Work}
\label{chapter6}

\section{Conclusions}

\section{Future Work}

\subsection{Terrain Generation Improvements}
% 目前只生成了洞穴, 并未生成地表, 未实现地表和洞穴的过渡, 未实现地表的生物群系, 未实现地表的植被, 未实现地表的水体

\subsection{Performance Optimization}
% 目前的Neighbor-face culling是以chunk为单位的, 区块与区块之间的面并未进行剔除, 可以尝试实现区块与区块之间的面剔除, 以提高性能

\subsection{Chunk file storage}
%目前的区块是在一个文件夹里. 每生成一个区块就会生成一个文件, 这样会导致文件过多, 可以尝试将区块存储在一个文件里, 以减少文件数量

\subsection{Multi-threading Optimization} 
% 目前的多线程不够完善, 仍然会对渲染循环造成一定的影响, 可以尝试优化多线程, 以减少对渲染循环的影响