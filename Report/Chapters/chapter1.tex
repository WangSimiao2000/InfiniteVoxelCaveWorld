\chapter{Introduction}
\label{chapter1}

% 项目的背景
\section{Context}

In recent years, procedural generation has become a fundamental technique in the development of virtual environments, particularly within the gaming industry. 
The application of procedural generation techniques enables the creation of expansive, complex, and meticulously detailed virtual environments with minimal manual input, thereby offering players distinctive and immersive experiences with each engagement with the content.

Notable examples include the video games \textit{Minecraft} and \textit{Terraria}. 
\textit{Minecraft}, developed by Mojang Studios, employs procedural generation to create an infinite world comprising a variety of terrains, including mountains, caves, forests, and oceans. 
Similarly, \textit{Terraria}, developed by Re-Logic, employs procedural generation to construct a two-dimensional world filled with caves, dungeons, and diverse biomes. 
The randomness and variety inherent in the generated worlds contribute significantly to the game's replayability, ensuring that each player's adventure is unique.

% 项目的问题陈述
\section{Problem statement}

The primary objective of this project is to develop a dynamic voxel cave terrain generation system using C++ and OpenGL. This system faces several challenges:

Dynamic loading and unloading of terrain chunks is a major challenge, requiring real-time generation and removal as the player moves through the environment. Efficient management of these chunks is crucial, involving their creation, updating, and removal without impacting performance. Multi-threading is used to handle terrain generation in parallel, reducing the load on the main thread.

Another key aspect is using noise functions for terrain generation. Effectively blending Perlin and Simplex noise to create realistic cave structures, and dynamically adjusting their influence, is essential for achieving desired results.

Performance optimization is critical, with techniques such as removing invisible voxel faces and excluding out-of-view chunks necessary to maintain high frame rates. Efficient implementation of these techniques is vital.

% 项目的目的和目标
\section{Aim and objectives}

The aim of this project is to develop a dynamic voxel cave terrain generation system using C++ and OpenGL. The system will create and manage cave structures using Perlin and Simplex noise. Users can control the camera using mouse and keyboard to navigate the scene, view the generated cave terrain from different angles, and modify noise weights and control terrain generation status through the user interface.

The main objectives for completing this project are:
\begin{enumerate}
    \item Develop a dynamic voxel terrain generation system that uses Perlin and Simplex noise to create cave structures. This includes implementing algorithms for generating terrain chunks based on noise values.
    \item Implement chunk management for real-time loading and unloading of terrain. The system should handle the creation, updating, and removal of chunks while minimizing impact on overall performance. Multi-threading will be employed to manage terrain generation concurrently with rendering.
    \item Optimize performance by applying techniques such as voxel face culling and frustum culling. Ensure high frame rates and smooth performance through effective removal of redundant and invisible voxel data.
    \item Create a user interface with ImGui to allow real-time adjustments of terrain generation parameters, including noise function weights, and to display performance metrics such as frame rate.
    \item Ensure seamless integration of system components by enabling real-time changes to terrain generation settings, supporting functionality to start/stop chunk loading, and providing the option to reset the camera position.
\end{enumerate}

% 项目的交付物
\section{Deliverables}

The expected deliverables are as follows:
\begin{enumerate}
    \item A dynamic voxel cave generation system developed with C++ and OpenGL. It uses Perlin and Simplex noise to create voxel-based cave systems, supports real-time chunk loading and unloading, and optimizes performance with techniques like frustum culling and voxel face removal. The system includes a user interface for real-time adjustments and performance monitoring.
    \item A GitLab repository that contains the source code of the system.
    \item Developer documentation.
    \item The MSc project report.
\end{enumerate}